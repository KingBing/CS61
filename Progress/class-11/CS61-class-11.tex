\documentclass[a4paper, 11pt]{article}

%%%%%% use packages %%%%%%
\usepackage{CJKutf8}
\usepackage{graphicx}
\usepackage[unicode,pdftex]{hyperref}
\usepackage{indentfirst}
\usepackage{latexsym}
\usepackage{amsmath}
\usepackage{flafter}
\usepackage{enumerate}
\usepackage{leftidx}

%%%%%% indent settings %%%%%%
\setlength{\parindent}{20pt}
\setlength{\parskip}{1ex plus 0.5ex minus 0.25ex}
\setlength{\baselineskip}{1.0ex}
\renewcommand{\baselinestretch}{1.5}

\begin{CJK}{UTF8}{gbsn}

\title{CS61第十一次课程记录}
\author{傅海平\\
\textsc{Institute Of Computing Technology,}\\
\textsc{Chinese Academy Of Sciences}\\
haipingf@gmail.com\\
}
\date{\today}
\begin{document}
\maketitle
\newpage
\tableofcontents
\newpage
\section{Topics}
\begin{center}
  \Large{Semaphores}
\end{center}

\section{progress}
早上8:30点开始,8:30 - 10:30 学习 lec19.pdf 课程讲义,然后11:00开始讨论学习过程中遇到的问题。
\section{learning details}
\subsection{course sketch}
\subsubsection{Semaphores}
\begin{itemize}
  \item{信号量}
	\begin{itemize}
	  \item{高层次同步}
	  \item{共享的计数器}
	  \item{PV操作,或wait()、signal(),或down()、up()}
	  \item{原子性}
	  \item{二元信号量}
	  \item{信号量的实现}
	  \item{生产者\&消费者问题}
	  \item{又称有界缓冲区模型}
		\subitem{1.\-}{生产者向缓冲区放入产品}
		\subitem{2.\-}{消费者从缓冲区中取出产品并进行消费}
		\subitem{3.\-}{当缓冲区为满时生产者必须等待}
		\subitem{4.\-}{当缓冲区为空时消费者必须等待}
	  \item{信号量函数库(不是pthread库的一部分)}
		\subitem{sem\_init()}
		\subitem{sem\_wait()}
		\subitem{sem\_post()}
		\subitem{sem\_getvalue()}
	  \item{信号量缺点}
	  \end{itemize}
  \end{itemize}
\subsubsection{Condition variables}
\begin{itemize}
  \item{条件变量}
  \item{条件变量:线程可以等待和唤醒的条件}
  \item{条件变量代表某一事件,不能从一次CV操作中保存或获取其值。}
  \item{三种操作}
	\subitem{wait()}
	\subitem{signal()}
	\subitem{broadcast()}
  \item{在Pthread中,条件变量表示为:pthread\_cond\_t}
	\subitem{pthread\_cond\_t cond = PTHREAD\_COND\_INITIALIZER;}
    \subitem{int pthread\_cond\_init(pthread\_cond\_t *cond, pthread\_condattr\_t
	 *cond\_attr);}
	 \subitem{int pthread\_cond\_signal(pthread\_cond\_t *cond);}
	 \subitem{int pthread\_cond\_broadcast(pthread\_cond\_t *cond);}
	 \subitem{int pthread\_cond\_wait(pthread\_cond\_t *cond, pthread\_mutex\_t *mutex);}
     \subitem{int pthread\_cond\_timedwait(pthread\_cond\_t   *cond,
	 pthread\_mutex\_t
	   *mutex, const struct timespec *abstime);}
	 \subitem{int pthread\_cond\_destroy(pthread\_cond\_t *cond);}
   \item{在实际编程时,条件变量应该由锁来保护}
	 \subitem{在某一条件变量上调用wait()将释放锁。}
	 \subitem{在某一条件变量上调用signal()会唤醒等待该条件变量的线程。}
	 \subitem{被唤醒的线程必须在wait()返回时重新申请锁。}
  \end{itemize}
\subsubsection{Monitors}
\begin{itemize}
  \item{监视器:Monitors}
  \item{监视器是一个包含了变量,条件变量和方法的对象。}
  \item{任何时刻只有一个线程在监视器中是活动的。}
  \end{itemize}
\subsection{Problems}
\subsection{Solutions}
\subsubsection{Posix Semaphores}
\begin{itemize}
  \item{int sem\_init(sem\_t *sem, int pshared, unsigned int value);}
  \item{int sem\_wait(sem\_t *sem);}
  \item{int sem\_post(sem\_t *sem);}
  \item{int sem\_getvalue(sem\_t *sem, int *valp);}
  \item{ int sem\_destroy(sem\_t *sem);}
  \end{itemize}
\section{测试}
\newpage
\end{CJK}
\end{document}
