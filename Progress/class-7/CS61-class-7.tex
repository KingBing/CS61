\documentclass[a4paper, 11pt]{article}

%%%%%% use packages %%%%%%
\usepackage{CJKutf8}
\usepackage{graphicx}
\usepackage[unicode]{hyperref}
\usepackage{indentfirst}
\usepackage{latexsym}
\usepackage{amsmath}
\usepackage{flafter}
\usepackage{enumerate}
\usepackage{leftidx}

%%%%%% indent settings %%%%%%
\setlength{\parindent}{20pt}
\setlength{\parskip}{1ex plus 0.5ex minus 0.25ex}
\setlength{\baselineskip}{1.0ex}
\renewcommand{\baselinestretch}{1.5}

\begin{CJK}{UTF8}{gbsn}


\title{CS61第七次课程记录}
\author{傅海平\\
\textsc{Institute Of Computing Technology,}\\
\textsc{Chinese Academy Of Sciences}\\
haipingf@gmail.com\\
}
\date{\today}
\begin{document}
\maketitle
\newpage
\tableofcontents
\newpage
\section{Topics}
\begin{center}
  \Large{Cache 性能测评和优化 \& 虚拟内存}
\end{center}

\section{Progress}
早上9点开始,9:00 - 10:50 学习 Lec14-Cache\_measurement.pdf 和
Lec15-Virtual\_Memory.pdf 两张课程讲义,然后11:00开始讨论学习过程中遇到的问题
。
\section{Learning Details}
\subsection{Course Sketch}
\subsubsection{Cache performance metrics}
\begin{itemize}
  \item{Miss Rate : 失效率, $L1: 3-10\%; L2: < 1\%$}
  \item{Hit Time  : 命中时间, 1-2 clock cycles for L1; 5-20 clock cycles for L2}
  \item{Miss Penalty : 失效损失, Typically 50-200 cycles for main memory}
  \item{平均访问时间 = $hit time + (miss rate \times miss penalty)$}
  \item{充分利用程序局部性:时间局部性和空间局部性}
  \item{如果事先不知道CPU的Cache指标,如何通过程序计算出CPU 的Cache 大小。}
	\begin{itemize}
	  \item{首先分配 $\omega$ 大小的数组}
	  \item{以$S$为步长重复访问内存元素,并计算每次访问时间}
	  \item{改变 $\omega$ 和 $S$,重复上述步骤,以此估算 Cache 的特性}
	  \end{itemize}
  \end{itemize}
\subsubsection{Discovering your cache's size and performance}
\subsubsection{Memory Mountain}
见讲义上的图。
\subsubsection{Matrix multiply, six ways}
\subsubsection{Blocked matrix multiplication}
\subsubsection{Exploiting locality in your programs}
\subsubsection{Running multiple programs at once}
\subsubsection{Virtual memory}
\subsection{Problems}
\subsection{Solutions}
\end{CJK}
\end{document}
